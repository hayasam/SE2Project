\pagebreak
\section{Functional requirements}
We divide the functional requirements of the application by the user class (actors) which are (mainly) involved in it:
\subsection{Passenger}
\begin{enumerate}
\item The system must not allow an already signed up Passenger to register himself (same username) again to the system
\item The username provided in the registration phase must not be empty
\item The password provided in the registration phase must not be empty
\item The system must notify the Passenger in case: 
	\begin{itemize}
	\item The username provided is empty
	\item The password provided is empty
	\item The username provided already exists in the system
	\end{itemize}
	and must abort the registration procedure
\item The system must provide for the Passenger a way to abort the registration procedure
\item If a passenger request for a taxi and his corresponding taxi queue contains waiting (available) taxis then the system must soon or later respond positively to the Passenger
\item A passenger request must be refuted if and only if there are no taxi driver available in the corresponding taxi queue or the number of passengers of the ride is greater than 3.
\item A reservation must be refused if and only if:
\begin{itemize}
	\item Origin and destination are the same location
	\item The number of passengers of the ride is greater than 3
	\item $\text{time}(\text{meeting time}) - \text{time}(\text{reservation}) < \text{2 hours}$
	\item $\text{time}(\text{meeting time}) \leq \text{time}(\text{reservation})$
\end{itemize}
\end{enumerate}


\subsection{Taxi driver}
\begin{enumerate}
\item A Taxi Driver, when he is sets himself as available, must be put at the bottom of the taxi queue relative to his corresponding taxi zone.
\item A Taxi Driver, when he is available, must be in one and exactly one taxi queue.
\item A Taxi Driver, when he is not available, must not be in any taxi queue.
\item At each position of the Taxi Driver retrieved from the GPS data must correspond exactly one and only one taxi zone
\item A taxi queue must have a number $n$ of taxi driver waiting in the range $n\in[0,+\infty[$
\item A Taxi driver must be always in one of this three states, which are mutually exclusive:
\begin{itemize}
	\item Available (see the glossary at Table \ref{glossary})
	\item Busy (see the glossary at Table \ref{glossary})
	\item Not available	(see the glossary at Table \ref{glossary})
\end{itemize}
\item The system must put the taxi driver at the bottom of the queue if:
\begin{itemize}
	\item He refuses a request
	\item He does not respond to a request (accepting or refusing) within a certain time from the reception of it (10 seconds).
\end{itemize}
\item A taxi driver can receive requests if and only if he is at the top of a taxi queue.
\item A taxi driver waiting (available) in a taxi queue must receive requests only from passengers which specified a meeting point inside the correspondent taxi zone
\item When a Taxi driver accepts a request from a Passenger, he must be removed from the corresponding taxi queue and set as busy.
\item Once a taxi driver sets himself as not available and the system removes him from the queue, that taxi driver must not be in any other queue: only available taxi driver must be in a queue.
\end{enumerate}