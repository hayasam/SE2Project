\pagebreak
\section{External interface requirements}
\subsection{User interfaces}
Here we present the mock-ups of the application user interface.\\
We will present the user interface regarding only the two mobile applications (one for the Passengers and one for the Taxi Drivers).\\
The web application interface for the Passenger is a natural extension of the mobile one.
\\
\\
The user interface must be as simple as possible: both the users must be able to use it immediately after the installation.
\subsubsection{Home page for the Passenger}
\begin{figure}[H]
\centering
\includegraphics[scale=0.6]{Images/home_page}
\caption{Home page for the Passenger}
\end{figure}
This screen will present to the Passenger the possibility to 
\begin{itemize}
\item Make a request
\item Reserve a taxi for a future journey
\item Check his pending reservations
\end{itemize}

\subsubsection{Request a taxi}
\begin{figure}[H]
\centering
\includegraphics[scale=0.6]{Images/taxi_request}
\caption{Make a request page}
\end{figure}
This screen is reached by pressing the button "REQUEST A TAXI" from the Home Page. The user can:
\begin{itemize}
\item Insert his location
\item The number of passengers of the requested ride
\item Submit the request
\end{itemize} 
If one or more necessary information is not inserted (missing) the user will be notified with an appropriate pop-up.


\subsubsection{Reservation of a taxi}
\begin{figure}[H]
\centering
\includegraphics[scale=0.6]{Images/taxi_reservation}
\caption{Reservation of a taxi}
\end{figure}
This screen is reached by pressing the button "RESERVE A TAXI" from the Home Page. The user can:
\begin{itemize}
\item Specify his actual location
\item Specify the destination of the journey
\item Specify the date and hour of the meeting time
\item Specify the number of passengers
\item Submit the reservation request to the system
\end{itemize}
If one or more necessary information is not inserted (missing) the user will be notified with an appropriate pop-up.

\subsubsection{Reservations}
\begin{figure}[H]
	\centering
	\includegraphics[scale=0.6]{Images/reservations}
	\caption{Reservations page of a passenger}
\end{figure}
This screen is reached by pressing the button "RESERVATIONS" from the Home Page. The user can:
\begin{itemize}
	\item View all the reservations
	\item Cancel a reservation
\end{itemize}
The permission of cancelling a reservation will be granted only if this is done at least 10 minutes before the meeting time. A pop-up will notify the passenger of the result of the operation.

\subsubsection{Acknowledgement of a taxi request/reservation to a Passenger}
\begin{figure}[H]
\centering
\includegraphics[scale=0.6]{Images/taxi_coming}
\end{figure}
This screen is reached when a taxi is allocated to the Passenger request, both in case of request and reservation.\\
The screen will present:
\begin{itemize}
\item Taxi identification number
\item Waiting time
\item Resume of the submitted data by the Passenger (meeting location, ecc...)
\end{itemize}
\pagebreak

\subsubsection{Start working day for a Taxi Driver}
This is the first screen a Taxi Driver will see at the opening of the application:
\begin{figure}[H]
\centering
\includegraphics[scale=0.6]{Images/start_taxi_driver}
\caption{First screen of the application for Taxi Drivers}
\end{figure}

In this page the Taxi Driver can:
\begin{itemize}
	\item Set himself as available
\end{itemize}

\subsubsection{Taxi Driver's waiting for a requests}
\begin{figure}[H]
\centering
\includegraphics[scale=0.6]{Images/wait_taxi_driver}
\caption{Waiting screen for a taxi driver}
\end{figure}
This screen is reached by pressing on the "START" button on the previous screen.\\
Once a request is submitted to the taxi driver the application notifies him with a pop-up saying that there is a request for a ride in the current zone.\\
It is asked to the taxi driver if he wants to accept the request or not. \\
Before receiving a request, the taxi driver can press the "STOP" button if he wants to set himself as not available. This action will bring him to the home page.

\subsubsection{Communication of the location to a Taxi Driver}
\begin{figure}[H]
\centering
\includegraphics[scale=0.6]{Images/taxi_driver_busy}
\caption{Communication of the Passenger location to the Taxi Driver}
\end{figure}
This screen is reached by pressing the "YES" button on the pop-up displayed on the previous screen. It will be present a map indicating the position of the Passenger (meeting location). When the ride will be done the Taxi Driver will press the "END OF RIDE" button to notify his availability.

\pagebreak
\subsection{GUI state chart}
We can summarize the behavior of the user interface through a state chart in which every state represents a specific screen of the application and each transition is basically a user input or a system communication to the application.
\subsubsection{Taxi driver usage of the user interface}
\begin{figure}[H]
\centering
\includegraphics[scale=0.45]{Images/statechart_GUI}
\end{figure}

\subsubsection{Passenger usage of the user interface}
\begin{figure}[H]
\centering
\includegraphics[scale=0.5]{Images/statechart_GUI_Passenger}
\end{figure}