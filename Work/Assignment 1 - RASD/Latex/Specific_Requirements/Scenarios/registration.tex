\subsection{Registration to the service}
\subsubsection{Scenario}
Bob has discovered the new service offered by the city called \textit{myTaxiService} and he wants to discover how it works. He goes to the web application site and starts the registration phase.\\
He insert a the username and the password he wants and submit the request of registration. The system replies saying that the username selected is already used and that Bob has to insert another username. Bob insert another username and this time the procedure succeed. Bob is now correctly registered to \textit{myTaxiService}.

\subsubsection{Use case}
\begin{center}
\centering
\begin{longtable}{| p{.20\textwidth} | p{.80\textwidth} |} \hline
Use case & \textbf{Register} \\ \hline 
Actors & Passenger \\ \hline
Goals & A passenger must be able to register to the service \\ \hline
Enter condition & None \\ \hline
Event flow & \begin{enumerate}
				\item The passenger goes to the web application page
				\item The passenger clicks on the sign up button
				\item \label{fillForm1} The passenger fills the form with username and password desired
				\item The passenger clicks the submit button
				\item If the username is already present in the system or there are missing data
				\begin{enumerate}
					\item Notify the Passenger of the error
					\item Go back to Event flow \ref{fillForm1}
				\end{enumerate}	
				\item The system retrieves the data and stores them
				\item The system notifies that the registration has been correctly done
			\end{enumerate} \\ \hline
Exceptions & If the user, during the insertion of the registration data (username + password) decides to abort the procedure by clicking on the proper button
			\begin{enumerate}
				\item The system notifies the passenger of the loss of the inserted data
				\item The system abort the procedure
			\end{enumerate}\\ \hline
Exit condition & The passenger is correctly registered to the service \\ \hline
\caption{Use case: Register}
\end{longtable}
\end{center}
