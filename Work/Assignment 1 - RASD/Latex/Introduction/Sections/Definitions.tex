\section{Definitions}
We present here the main glossary of the application domain, derived from the client specification already reported: \\ \\
\begin{center}
\begin{longtable}{| p{.20\textwidth} | p{.80\textwidth} |} \hline
{\Large \textbf{Term}} & {\Large \textbf{Description}} \\ \hline
Available & Status of a Taxi Driver in which he can receive requests \\ \hline
Busy & Status of a Taxi Driver in which he is serving a Passenger's request \\ \hline
Government of the city & It is the client for which we are working.  It desires an application for the improvement and simplification of the taxi service. \\ \hline
City & The ambient in which the taxi drivers and the passengers interact. It is divided in taxi zones. \\ \hline
GPS & Technology which manage to get in every moment the position of a vehicle \\ \hline
Meeting point/location & It is the location inside the city at which the passenger and the taxi driver will meet. It is set by the Passenger during the reservation procedure \\ \hline
Meeting time & It is the time at which the passenger and the taxi driver will meet. It is set by the Passenger during the reservation and request procedures\\ \hline
Mobile application & It is one of the interface that passengers and taxi drivers can use to interact with the system. To use it they must have it installed in their smart phone. \\ \hline
Not available & Status of a Taxi Driver in which he cannot receive requests \\ \hline
Passenger & One of the user of the system. He can apply for a taxi service: he can make a request or a reservation\\ \hline
Programmatic interface & It is a software interface to be used by developers to modify and extend the actual software. It is useful for the extension of the application with additional taxi services \\ \hline
Queue management & It is the part of the system that takes care of the organization of the taxi drivers inside the taxi queue of the correspondent taxi zone \\ \hline
Request & It is the action carried out by the passenger when he needs to use the taxi service. It represents an immediate need of the passenger \\ \hline
Reservation & It is the action carried out by the passenger when he needs to use the taxi service in the future. It consists in the specification of the origin and the destination of the taxi ride, reserved at a desired time. The passenger can reserve a journey only if the specified time is at least two hours after the time of reservation. \\ \hline
System & It is the application we have to design. It is constituted by different interfaces: two for the Passengers, one for the Taxi Drivers and one for the future extension of the application (programmatic interface). \\ \hline
Taxi driver & One of the user of the system. They are the people which task is to drive the taxi. They use the application to communicate their availability or not and to be assigned to the taxi queues. \\ \hline
Taxi queue & It is an abstract queue of the available taxis in a taxi zone of the city. It has an order determined by the time in which each taxi driver communicate its availability to the system or by the decision of the taxi driver to accept a request or not \\ \hline
Taxi service & It is a service offered by the government of the large city which enable every person to use the offered taxis. It can be of different nature as a reservation service or a request. \\ \hline
Taxi zone & The city is divided in taxi zones, each one having its taxi queue. The division is provided in order to distribute the taxi availability in all the city territory \\ \hline
Web application & Is one of the interfaces that the passengers can use to interact with the system. To use it the passenger must use a web browser. \\ \hline
\caption{Glossary}
\label{glossary}
\end{longtable}
\end{center}
