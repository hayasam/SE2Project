\chapter{Introduction}

\section{Purpose}
This is the Requirements Analysis and Specification Document to be used in the design of the software system called \textit{myTaxiService}. In this document we describe the specifications and constraints that the software we are to implement must have.
The intended audience of this paper is:
\begin{itemize}
\item The project manager
\item The client which in this case is the Government of the city
\item The designers and developers of the application
\item The testing team
\item The end user
\end{itemize}
This document has contractual value.
\section{Scope}
Here we summarize the main scope of the application.\\
The client is the \textit{government of a large city}.\\
This city already offers a taxi service to its citizens, but the government wants to improve it using a modern and efficient information system.\\
So we received the request to design and implement this application, called \textit{myTaxiService} which has basically two great objectives:
\begin{itemize}
\item Simplify the usage of the taxi service
\item Guarantee an efficient management of the taxi queues
\end{itemize}
\subsection{Users and interfaces of the application}
The system is designed to interact with two kind of users:
\begin{itemize}
\item The clients (Passengers) of the Taxi service
\item The taxi drivers
\end{itemize}
For each category of users we must provide an appropriate interface and the client requests for the interfaces are:
\begin{itemize}
\item A web application or a mobile application for the clients (Passengers)
\item A mobile application for the Taxi Drivers
\end{itemize}
Moreover, it is required to implement a software interface for the developers of the system that manages to simplify the extension of the software with additional taxi services.
\subsection{High level behavior}
Here we report the exact description of the problem the government of the city supplied to us: \\ \\ 
Passengers can request a taxi either through a web application or a mobile application.
The system answers to the request by informing the passenger about the code of the incoming taxi and the waiting time.\\ \\
Taxi drivers use a mobile application to inform the system about their availability and to confirm that they are going to take care of a certain request. \\ \\  
The system guarantees a fair management of taxi queues. \\
In particular the city is divided in taxi zones (approximately 2 km2 each). Each taxi zone is associated to a taxi queue.\\
The system automatically computes the distribution of taxi in the various taxi zones based on the GPS information it receives from each taxi. When a taxi is available, its identifier is stored in the taxi queue in the corresponding taxi zone. When a request arrives from a certain taxi zone, the system forwards it to the first taxi queuing (in the taxi queue) in that taxi zone. If the taxi confirms (the request), then the system will send a confirmation to the passenger. If not, then the system will forward the request to the second in the taxi queue and will, at the same time, move the first taxi in the last position in the taxi queue.\\ \\
A passenger can reserve a taxi by specifying the origin and the destination of the ride. The reservation has to occur at least two hours before the ride. In this case, the system confirms the reservation to the passenger and allocates a taxi to the reservation 10 minutes before the meeting time with the passenger. \\ \\
Beside the specific user interfaces for passengers and taxi drivers, the system offers also programmatic interfaces to enable the development of additional taxi service (e.g. taxi sharing) on top of the basic one. 

\section{Definitions}
We present here the main glossary of the application domain, derived from the client specification already reported: \\ \\
\begin{center}
\begin{longtable}{| p{.20\textwidth} | p{.80\textwidth} |} \hline
{\Large \textbf{Term}} & {\Large \textbf{Description}} \\ \hline
Available & Status of a Taxi Driver in which he can receive requests \\ \hline
Busy & Status of a Taxi Driver in which he is serving a Passenger's request \\ \hline
Government of the city & It is the client for which we are working.  It desires an application for the improvement and simplification of the taxi service. \\ \hline
City & The ambient in which the taxi drivers and the passengers interact. It is divided in taxi zones. \\ \hline
GPS & Technology which manage to get in every moment the position of a vehicle \\ \hline
Meeting point/location & It is the location inside the city at which the passenger and the taxi driver will meet. It is set by the Passenger during the reservation procedure \\ \hline
Meeting time & It is the time at which the passenger and the taxi driver will meet. It is set by the Passenger during the reservation and request procedures\\ \hline
Mobile application & It is one of the interface that passengers and taxi drivers can use to interact with the system. To use it they must have it installed in their smart phone. \\ \hline
Not available & Status of a Taxi Driver in which he cannot receive requests \\ \hline
Passenger & One of the user of the system. He can apply for a taxi service: he can make a request or a reservation\\ \hline
Programmatic interface & It is a software interface to be used by developers to modify and extend the actual software. It is useful for the extension of the application with additional taxi services \\ \hline
Queue management & It is the part of the system that takes care of the organization of the taxi drivers inside the taxi queue of the correspondent taxi zone \\ \hline
Request & It is the action carried out by the passenger when he needs to use the taxi service. It represents an immediate need of the passenger \\ \hline
Reservation & It is the action carried out by the passenger when he needs to use the taxi service in the future. It consists in the specification of the origin and the destination of the taxi ride, reserved at a desired time. The passenger can reserve a journey only if the specified time is at least two hours after the time of reservation. \\ \hline
System & It is the application we have to design. It is constituted by different interfaces: two for the Passengers, one for the Taxi Drivers and one for the future extension of the application (programmatic interface). \\ \hline
Taxi driver & One of the user of the system. They are the people which task is to drive the taxi. They use the application to communicate their availability or not and to be assigned to the taxi queues. \\ \hline
Taxi queue & It is an abstract queue of the available taxis in a taxi zone of the city. It has an order determined by the time in which each taxi driver communicate its availability to the system or by the decision of the taxi driver to accept a request or not \\ \hline
Taxi service & It is a service offered by the government of the large city which enable every person to use the offered taxis. It can be of different nature as a reservation service or a request. \\ \hline
Taxi zone & The city is divided in taxi zones, each one having its taxi queue. The division is provided in order to distribute the taxi availability in all the city territory \\ \hline
Web application & Is one of the interfaces that the passengers can use to interact with the system. To use it the passenger must use a web browser. \\ \hline
\caption{Glossary}
\label{glossary}
\end{longtable}
\end{center}

\section{References}
\begin{itemize}
\item IEEE Std 830-1998: \textit{IEEE Recommended Practice for Software Requirements Specifications}
\item Assignment 1 document
\end{itemize}
\section{Overview}
In the next sessions of this document we will discuss about:
\begin{enumerate}
\item (Chapter 2) \textbf{Overall Description}
\item (Chapter 3) \textbf{Specific Requirements}
\end{enumerate}
