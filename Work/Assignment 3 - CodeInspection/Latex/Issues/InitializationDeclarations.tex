\subsection{Initialization and Declarations}
\begin{enumerate}
\setcounter{enumi}{27}
	\item Check that variables and class members are of the correct type. Check that they have the right visibility (public/private/protected)
	\begin{itemize}
	 	\item Method 1: \cmark
 		\item Method 2: \cmark
	\end{itemize}
	\item Check that variables are declared in the proper scope
	\begin{itemize}
	 	\item Method 1: \cmark
 		\item Method 2: \cmark
	\end{itemize}
	\item Check that constructors are called when a new object is desired
	\begin{itemize}
	 	\item Method 1: \cmark
 		\item Method 2: \cmark
	\end{itemize}
	\item Check that all object references are initialized before use
	\begin{itemize}
	 	\item Method 1: \cmark
 		\item Method 2: \cmark
	\end{itemize}
	\item Variables are initialized where they are declared, unless dependent upon a computation
	\begin{itemize}
	 	\item Method 1: \cmark
 		\item Method 2: \cmark
	\end{itemize}
	\item Declarations appear at the beginning of blocks (A block is any code surrounded by curly braces '\{' and '\}' ). The exception is a variable can be declared in a \texttt{for} loop
	\begin{itemize}
	 	\item Method 1: \cmark
 		\item Method 2: \xmark\\
 		At line 166, inside an if block, a variable is declared after the assignment of another one.
 		\code{162}{167}
	\end{itemize}
\end{enumerate}