\subsection{Naming Conventions}
\begin{enumerate}
\setcounter{enumi}{0}
	\item All class names, interface names, method names, class variables, method variables, and constants used should have meaningful names and do what the name suggests:
	\begin{itemize}
	 	\item Method 1: \cmark
		 \item Method 2: 
	\end{itemize}
	\item If one-character variables are used, they are used only for temporary "throwaway” variables, such as those used in for loops.
	\begin{itemize}
	 	\item Method 1: \cmark
	 \item Method 2: 
	\end{itemize}
	\item Class names are nouns, in mixed case, with the first letter of each word in capitalized.
	\begin{itemize}
	 	\item Method 1: \cmark
 		\item Method 2: 
	\end{itemize}
	\item Interface names should be capitalized like classes
	\begin{itemize}
	 	\item Method 1: \cmark
 		\item Method 2: 
	\end{itemize}	
	\item Method names should be verbs, with the first letter of each addition word capitalized.
	\begin{itemize}
	 	\item Method 1: \cmark
 		\item Method 2: 
	\end{itemize}
	\item Class variables, also called attributes, are mixed case, but might begin with an underscore
	('\_') followed by a lowercase first letter. All the remaining words in the variable name have
	their first letter capitalized
	\begin{itemize}
	 	\item Method 1: \cmark
 		\item Method 2: 
	\end{itemize}
	\item Constants are declared using all uppercase with words separated by an underscore
	\begin{itemize}
	 	\item Method 1: \cmark
 		\item Method 2: 
	\end{itemize}	
\end{enumerate}