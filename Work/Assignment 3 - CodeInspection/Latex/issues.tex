\section{Issues found by applying the checklist}
We use the following notation:
\begin{itemize}
	\item \cmark: the relative point in the checklist is satisfied by the method
	\item \xmark: the relative point in the checklist is not satisfied and will follow the piece of code affected
	by the problem or a description of the problem
\end{itemize}


\subsection{Naming Conventions}
\begin{enumerate}
\setcounter{enumi}{0}
	\item All class names, interface names, method names, class variables, method variables, and constants used should have meaningful names and do what the name suggests:
	\begin{itemize}
	 	\item Method 1: \cmark
		 \item Method 2: 
	\end{itemize}
	\item If one-character variables are used, they are used only for temporary "throwaway” variables, such as those used in for loops.
	\begin{itemize}
	 	\item Method 1: \cmark
	 \item Method 2: 
	\end{itemize}
	\item Class names are nouns, in mixed case, with the first letter of each word in capitalized.
	\begin{itemize}
	 	\item Method 1: \cmark
 		\item Method 2: 
	\end{itemize}
\end{enumerate}
\subsection{Indention}
\begin{enumerate}
	\setcounter{enumi}{7}
	\item Three or four spaces are used for indentation and done so consistently:
	\begin{itemize}
		\item Method 1: \cmark
		\item Method 2:
	\end{itemize}
	\item No tabs are used to indent:
	\begin{itemize}
		\item Method 1: \cmark
		\item Method 2: 
	\end{itemize}
\end{enumerate}
\subsection{Braces}
\begin{enumerate}
	\setcounter{enumi}{9}
	\item Consistent bracing style is used, either the preferred “Allman” style (first brace goes underneath the opening block) or the “Kernighan and Ritchie” style (first brace is on the same line of the instruction that opens the new block) :
	\begin{itemize}
		\item Method 1: \xmark \\
		The bracing style used is not consistent: in the method declaration the first brace is underneath the opening block, whereas in the rest of the method is in the same line.
		\code{106}{109}
		\item Method 2:
	\end{itemize}
	\item All if, while, do-while, try-catch, and for statements that have only one statement to execute are surrounded by curly braces:
	\begin{itemize}
		\item Method 1: \cmark
		\item Method 2: 
	\end{itemize}
\end{enumerate}
\subsection{File organization}
\begin{enumerate}
	\setcounter{enumi}{11}
	\item Blank lines and optional comments are used to separate sections (beginning comments, package/import statements, class/interface declarations which include class variable/attributes declarations, constructors, and methods) :
	\begin{itemize}
		\item Method 1: \cmark
		\item Method 2: \cmark
	\end{itemize}
	\item Where practical, line length does not exceed 80 characters:\label{13}
	\begin{itemize}
		\item Method 1: \xmark \\
		Often in the code, lines exceed 80 characters.
		\code{111}{114}
		\item Method 2: \xmark\\
		Often in the code, lines exceed 80 characters.
		\code{171}{172}
		$$**$$
		\code{191}{192}
		$$**$$
		\code{235}{235}
		$$**$$
		\code{239}{239}
	\end{itemize}
	\item When line length must exceed 80 characters, it does NOT exceed 120 characters:
	\begin{itemize}
		\item Method 1: \xmark \\
		There is one line (ln.138) exceeding even 120 characters.
		\code{137}{140}
		\item Method 2: \xmark\\
		See the point \ref{13}
	\end{itemize}
\end{enumerate}
\subsection{Wrapping Lines}
\begin{enumerate}
	\setcounter{enumi}{14}
	\item Line break occurs after a comma or an operator :
	\begin{itemize}
		\item Method 1: \xmark \\
		This never happens. Not even in the method declaration.
		\code{106}{106}
		\item Method 2:
	\end{itemize}
	\item Higher-level breaks are used: ***** NOT SURE ABOUT THIS ONE *****
	\begin{itemize}
		\item Method 1:
		\item Method 2:
	\end{itemize}
	\item A new statement is aligned with the beginning of the expression at the same level as the previous line:
	\begin{itemize}
		\item Method 1: \cmark
		\item Method 2: 
	\end{itemize}
\end{enumerate}
\subsection{Comments}
\begin{enumerate}
\setcounter{enumi}{17}
	\item  Comments are used to adequately explain what the class, interface, methods, and blocks
	of code are doing.
		\begin{itemize}
		 	\item Class: \xmark\\
		 	Some methods do not present any comment above and it is necessary to make a reverse engineering of the code in order to understand what they do.
		\end{itemize}
	\item Commented out code contains a reason for being commented out and a date it can be removed from the source file if determined it is no longer needed.
		\begin{itemize}
		 	\item Method 1: \cmark
	 		\item Method 2: \cmark
		\end{itemize}
\end{enumerate}
\subsection{Java Source Files}
\begin{enumerate}
\setcounter{enumi}{19}
	\item Each Java source file contains a single public class or interface.
	\begin{itemize}
	 	\item Method 1: \cmark
 		\item Method 2: 
	\end{itemize}
	\item The public class is the first class or interface in the file.
	\begin{itemize}
	 	\item Method 1: \cmark
 		\item Method 2: 
	\end{itemize}
	\item Check that the external program interfaces are implemented consistently with what is described in the javadoc.
	\begin{itemize}
	 	\item Method 1: \cmark
 		\item Method 2: 
	\end{itemize}
	\item Check that the javadoc is complete
	\begin{itemize}
	 	\item Method 1: \xmark\\
	 	The Javadoc is not complete: it does not explain what this method is for and does not describe the kind and the role of the output of this method.
 		\item Method 2: 
	\end{itemize}
\end{enumerate}
\subsection{Package import statements}
\begin{enumerate}
\setcounter{enumi}{23}
	\item If any package statements are needed, they should be the first noncomment statements. Import statements follow.
	\begin{itemize}
	 	\item Class: \cmark
	\end{itemize}	
\end{enumerate}
\subsection{Class and Interface Declarations}
\begin{enumerate}
	\setcounter{enumi}{24}
	\item The class or interface declarations shall be in the following order :
	\begin{enumerate}[label=\Alph*.]
		\item class/interface documentation comment
		\item class or interface statement
		\item class/interface implementation comment, if necessary
		\item class (static) variables
		\begin{enumerate}[label=\alph*.]
			\item first public class variables
			\item next protected class variables
			\item next package level (no access modifier)
			\item last private class variables
		\end{enumerate}
		\item instance variables
		\begin{enumerate}[label=\alph*.]
			\item first public instance variables
			\item next protected instance variables
			\item next package level (no access modifier)
			\item last private instance variables
		\end{enumerate}
		\item constructors
		\item methods
	\end{enumerate}
	\begin{itemize}
		\item Class: \cmark
	\end{itemize}
	\item Methods are grouped by functionality rather than by scope or accessibility:
	\begin{itemize}
		\item Class: \cmark
	\end{itemize}
	\item Check that the code is free of duplicates, long methods, big classes, breaking encapsulation, as well as if coupling and cohesion are adequate:
	\begin{itemize}
		\item Class: \cmark
	\end{itemize}
\end{enumerate}
\subsection{Initialization and Declarations}
\begin{enumerate}
\setcounter{enumi}{27}
	\item Check that variables and class members are of the correct type. Check that they have the right visibility (public/private/protected)
	\begin{itemize}
	 	\item Method 1: \cmark
 		\item Method 2: 
	\end{itemize}
	\item Check that variables are declared in the proper scope
	\begin{itemize}
	 	\item Method 1: \cmark
 		\item Method 2: 
	\end{itemize}
	\item Check that constructors are called when a new object is desired
	\begin{itemize}
	 	\item Method 1: \cmark
 		\item Method 2: 
	\end{itemize}
	\item Check that all object references are initialized before use
	\begin{itemize}
	 	\item Method 1: \cmark
 		\item Method 2: 
	\end{itemize}
	\item Variables are initialized where they are declared, unless dependent upon a computation
	\begin{itemize}
	 	\item Method 1: \cmark
 		\item Method 2: 
	\end{itemize}
	\item Declarations appear at the beginning of blocks (A block is any code surrounded by curly braces '\{' and '\}' ). The exception is a variable can be declared in a \texttt{for} loop
	\begin{itemize}
	 	\item Method 1: \cmark
 		\item Method 2: 
	\end{itemize}
\end{enumerate}
\subsection{Method Calls}
\begin{enumerate}
	\setcounter{enumi}{33}
	\item Check that parameters are presented in the correct order :
	\begin{itemize}
		\item Method 1: \cmark
		\item Method 2:
	\end{itemize}
	\item Check that the correct method is being called, or should it be a different method with a similar name:
	\begin{itemize}
		\item Method 1: \cmark
		\item Method 2:
	\end{itemize}
	\item Check that method returned values are used properly:
	\begin{itemize}
		\item Method 1: \cmark
		\item Method 2: 
	\end{itemize}
\end{enumerate}
\subsection{Arrays}
\begin{enumerate}
	\setcounter{enumi}{36}
	\item Check that there are no off-by-one errors in array indexing (that is, all required array elements are correctly accessed through the index):
	\begin{itemize}
		\item Method 1: \cmark
		\item Method 2:
	\end{itemize}
	\item Check that all array (or other collection) indexes have been prevented from going out-of-bounds:
	\begin{itemize}
		\item Method 1: \cmark
		\item Method 2:
	\end{itemize}
	\item Check that constructors are called when a new array item is desired:
	\begin{itemize}
		\item Method 1: \cmark
		\item Method 2: 
	\end{itemize}
\end{enumerate}
\subsection{Object Comparisons}
\begin{enumerate}
\setcounter{enumi}{39}
	\item  Check that all objects (including Strings) are compared with "equals" and not with "=="
	\begin{itemize}
		\item Method 1: \cmark
		\item Method 2: \cmark
	\end{itemize}
\end{enumerate}
\subsection{Output format}
\begin{enumerate}
	\setcounter{enumi}{40}
	\item Check that displayed output is free of spelling and grammatical errors:
	\begin{itemize}
		\item Method 1: \cmark
		\item Method 2:
	\end{itemize}
	\item Check that error messages are comprehensive and provide guidance as to how to correct the problem:
	\begin{itemize}
		\item Method 1: \cmark
		\item Method 2:
	\end{itemize}
	\item Check that the output is formatted correctly in terms of line stepping and spacing:
	\begin{itemize}
		\item Method 1: \cmark
		\item Method 2: 
	\end{itemize}
\end{enumerate}
\subsection{Computation, Comparisons and Assignments}
\subsection{Exceptions}
\begin{enumerate}
	\setcounter{enumi}{51}
	\item Check that the relevant exceptions are caught
	\begin{itemize}
		\item Method 1: \cmark
		\item Method 2: \cmark
	\end{itemize}
	\item Check that the appropriate action are taken for each catch block
	\begin{itemize}
		\item Method 1: \cmark
		\item Method 2: \cmark
	\end{itemize}
\end{enumerate}
\subsection{Flow of control}
\begin{enumerate}
\setcounter{enumi}{53}
	\item In a switch statement, check that all cases are addressed by break or return
	\begin{itemize}
		\item Method 1: \cmark
		\item Method 2:
	\end{itemize}
	\item Check that all switch statements have a default branch
	\begin{itemize}
		\item Method 1: \cmark
		\item Method 2:
	\end{itemize}
	\item Check that all loops are correctly formed, with the appropriate initialization, increment and termination expressions
	\begin{itemize}
		\item Method 1: \cmark
		\item Method 2:
	\end{itemize}
\end{enumerate}
\subsection{Files}
\begin{enumerate}
	\setcounter{enumi}{56}
	\item Check that all files are properly declared and opened
	\begin{itemize}
		\item Method 1: \cmark
		\item Method 2:
	\end{itemize}
	\item Check that all files are closed properly, even in the case of an error
	\begin{itemize}
		\item Method 1: \cmark
		\item Method 2:
	\end{itemize}
	\item Check that EOF conditions are detected and handled correctly
	\begin{itemize}
		\item Method 1: \cmark
		\item Method 2:
	\end{itemize}
	\item Check that all file exceptions are caught and dealt with accordingly
	\begin{itemize}
		\item Method 1: \cmark
		\item Method 2:
	\end{itemize}
\end{enumerate}


