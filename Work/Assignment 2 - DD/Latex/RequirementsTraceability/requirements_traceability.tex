\chapter{Requirements Traceability}\label{chapter:requirementsTraceability}
\begin{center}
\begin{longtable}{|p{0.1\textwidth}|p{0.3\textwidth}|>{\raggedright\arraybackslash}p{0.2\textwidth}|>{\raggedright\arraybackslash}p{0.4\textwidth}|}
\hline
\multicolumn{2}{|c|}{\textbf{Requirement}} & \multicolumn{2}{c|}{\textbf{Design Solution}} \\ \hline
R.P.1 & Passengers can't register twice with same username & CMP: \linebreak \hyperref[comp:accountManager]{Account Manager} & The AccountManager manages the login and registration phase, avoiding this to happen  \\ \hline
R.P.2 & Username provided by passenger in registration phase must not be empty & CMP: \linebreak \hyperref[comp:accountManager]{Account Manager} & It makes this check \\ \hline
R.P.3 & The password provided in the registration phase must not be empty & CMP: \linebreak \hyperref[comp:accountManager]{Account Manager} & It makes this check \\ \hline
R.P.4 & The system must notify and abort the registration procedure in the previous 3 cases & CMP: \linebreak \hyperref[comp:accountManager]{Account Manager} & The $register(username,password)$ method provided by the Account Manager will return an error in any of the cases and the passenger GUI will show an appropriate pop-up. \\ \hline
R.P.5 & The system must provide for the Passenger with a way to abort the registration procedure & UX:\linebreak \hyperref[ux:passengerApp]{Passenger App} & Will be provided, in the user interface, a specific button for stopping the registration procedure \\ \hline
R.P.6\label{R.P.6} & If a passenger makes a request and the corresponding taxi queue is not empty, then the system must sooner or later respond positively to the Passenger &
SD: \linebreak\hyperref[seq:passengerMakesRequest]{Passenger request}\linebreak \linebreak
\hyperref[seq:provisionOfATaxi]{Provision of a taxi}
 & The TaxiManager will return a taxi to who wants it if the corresponding queue is not empty, and an error message if empty \\ \hline
R.P.7 & A passenger request must be refused if and only if there are no taxi driver available in the corresponding taxi queue or the number of passengers of the ride is greater than 3 & CMP: \linebreak \hyperref[comp:rideManager]{RideManager} & The  RequestHandler inside the RideManager will use its parser and will return an error in case the specified number of passengers is greater than 3. For what the emptiness of the queue refer to \hyperref[R.P.6]{R.P.6} \\ \hline
R.P.8\label{R.P.8} & A reservation must be refused if:
\begin{itemize}
	\item Origin and destination are the same location
	\item The number of passengers of the ride is greater than 3
	\item time(meeting time) - time(reservation) $<$ 2 hours
\end{itemize} & CMP:\linebreak \hyperref[comp:rideManager]{RideManager} \linebreak SD: \linebreak \hyperref[seq:passengerMakesReservation]{Reservation} & The  ReservationHandler inside the RideManager will use its parser and will return an error in case the specified number of passengers is greater than 3 or the source and destination are the same. It will cover the possibility to retry, in case the corresponding queue is empty, to forward the request. This is done until a taxi is found\\ \hline
R.P.9 & When the system looks for a taxi driver for serving a reservation (10 minutes before the meeting time), if the corresponding taxi queue is empty, it will wait until there is at least one taxi driver in the queue. & CMP:\linebreak \hyperref[comp:rideManager]{RideManager} \linebreak SD: \linebreak \hyperref[seq:passengerMakesReservation]{Reservation} & Refer to \hyperref[R.P.8]{R.P.8}\\ \hhline{|=|=|=|=|}
R.T.1 & A Taxi Driver, when he sets himself as available, must be put at the bottom of the taxi queue relative to his taxi zone. & CMP: \linebreak \hyperref[comp:queueManager]{QueueManager} \linebreak \hyperref[comp:taxiManager]{TaxiManager} \linebreak
ALG: \linebreak \hyperref[alg:addingOfATaxi]{Adding of a taxi} & The queues inside QueueContainer subcomponent of the QueueManager are managed as a FIFO structure\\ \hline 
R.T.2\label{R.T.2} & A Taxi Driver, when he is available, must be in one and exactly one taxi queue. & CMP: \linebreak \hyperref[comp:queueManager]{QueueManager} \linebreak \hyperref[comp:taxiManager]{TaxiManager} & The TaxiManager checks the availability status of the taxi driver and puts him in exactly one queue, if he is available \\ \hline
R.T.3 & A Taxi Driver, when he is not available, must not be in any taxi queue. & CMP: \linebreak \hyperref[comp:queueManager]{QueueManager} \linebreak \hyperref[comp:taxiManager]{TaxiManager} & Refer to \hyperref[R.T.2]{R.T.2}  \\ \hline
R.T.4 \label{R.T.4} & At each position of the Taxi Driver retrieved from the GPS data must correspond exactly one and only one taxi zone & CMP:\linebreak \hyperref[comp:geographicEngine]{GeographicEngine} & This component has the functionality to associate to each location exactly one taxi zone \\ \hline
R.T.5 & A taxi queue must have a number n of taxi driver waiting in the range $n \in [0, +\infty[$ & CMP: \linebreak \hyperref[comp:queueManager]{QueueManager} \linebreak \hyperref[comp:taxiManager]{TaxiManager} \linebreak ALG: \linebreak
\hyperref[alg:addingOfATaxi]{Adding of a taxi} & The structure of the queue object is a FIFO dynamic structure \\ \hline
R.T.6 & A Taxi driver must be always in one of this three states, which are mutually exclusive: available, busy, not available & CMP: \linebreak \hyperref[comp:taxiManager]{TaxiManager} & The TaxiManager manages correctly the states of each taxi driver \\ \hline
R.T.7 & The system must put the taxi driver at the bottom of the queue if: he refuses a request, he does not respond to a request within a certain time from the reception of it (10 seconds) & CMP: \linebreak \hyperref[comp:taxiManager]{TaxiManager} \linebreak SD: \linebreak \hyperref[seq:provisionOfATaxi]{Provision of a taxi} & The TaxiManager component manages the message sending and receiving form the taxi driver. \\ \hline
R.T.8 & A taxi driver can receive requests if and only if he is at the top of a taxi queue & CMP: \linebreak \hyperref[comp:queueManager]{QueueManager} \linebreak ALG:  \linebreak \hyperref[alg:retrievingOfATaxi]{Retrieving of a taxi} & The FIFO structure of the queue object and the usage of this structure by the QueueManager makes this happen \\ \hline
R.T.9 & A taxi driver waiting (available) in a taxi queue must receive requests only from passengers which specified a meeting point inside the correspondent taxi zone & CMP: \linebreak \hyperref[comp:geographicEngine]{GeographicEngine} \linebreak \hyperref[comp:taxiManager]{TaxiManager} \linebreak \hyperref[comp:queueManager]{QueueManager} \linebreak ALG: \linebreak \hyperref[alg:retrievingOfATaxi]{Retrieving of a taxi} & Refer to \hyperref[R.T.4]{R.T.4} for the GeographicEngine. \linebreak The TaxiManager will use the GeographicEngine in order to retrieve the correct taxi zone. \linebreak The Queue manager will make the link between the taxi zone and the taxi queue  \\ \hline
R.T.10 & When a Taxi driver accepts a request from a Passenger, he must be removed from the corresponding taxi queue and set as busy. & CMP: \linebreak \hyperref[comp:taxiManager]{TaxiManager} \linebreak \hyperref[comp:queueManager]{QueueManager} & The TaxiManager, received the positive message from the taxi driver, will set his state as busy (it was already removed from the queue) \\ \hline
\caption{Requirements traceability table} 
\label{tab:reqTraceTable}
\end{longtable} 
\end{center}
\paragraph{} As far as concern the requirement of the programmatic interface, it was already explained at \hyperref[design:programmaticInterface]{the relative section}.