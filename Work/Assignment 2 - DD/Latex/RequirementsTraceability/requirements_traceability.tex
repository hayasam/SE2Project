\chapter{Requirements Traceability}\label{chapter:requirementsTraceability}
Here we summarize our design solutions and 
\begin{table}[H]
\centering
\begin{longtable}{|p{0.05\textwidth}|p{0.3\textwidth}|p{0.1\textwidth}|p{0.5\textwidth}|}
\hline
\multicolumn{2}{|c|}{\textbf{Requirement}} & \multicolumn{2}{c|}{\textbf{Design Solution}} \\ \hline
R.P.1 & Passengers can't register twice with same username & \hyperref[comp:accountManager]{Account Manager} & The AccountManager manages the login and registration phase, avoiding this to happen  \\ \hline
R.P.2 & Username provided by passenger in registration phase must not be empty & \hyperref[comp:accountManager]{Account Manager} & It makes this check \\ \hline
R.P.3 & The password provided in the registration phase must not be empty & \hyperref[comp:accountManager]{Account Manager} & It makes this check \\ \hline
R.P.4 & The system must notify and abort the registration procedure in the previous 3 cases & \hyperref[comp:accountManager]{Account Manager} & The method $register(username,password)$ provided by the Account Manager will return an error in any of the cases and the passenger GUI will show an appropriate pop-up. \\ \hline
R.P.5 & The system must provide for the Passenger with a way to abort the registration procedure & \hyperref[ux:passengerApp]{Passenger user interface} & Will be provided, in the user interface, a specific button for stopping the registration procedure \\ \hline
R.P.6 & If a passenger makes a request and the corresponding taxi queue is not empty, then the system must sooner or later respond positively to the Passenger & 
\hyperref[seq:passengerMakesRequest]{Passenger request sequence diagram} &  \\ \hline
\end{longtable} 
\caption{Requirements traceability table}
\label{tab:reqTraceTable}
\end{table}