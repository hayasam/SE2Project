\chapter{Algorithm Design}\label{chapter:algorithmDesign}
\paragraph{}In this section we clarify some fundamental algorithm used by our system in order to achieve its objectives. We will deal basically with high level interactions between the components described in the previous sections.\\
All the algorithms are presented without any reference to a particular programming language but we are assuming an object-oriented paradigm for it.
\paragraph{}Since all the architecture is designed for a client - server interaction an obvious choice of framework would be the \textit{Java Enterprise Edition} framework and so the usage of the Java programming language.

\section{Queue management}\label{alg:queueManagement}
\paragraph{}The following algorithm is in charge of the management of the different taxi queues. It will be used by the \textbf{ManagementLogic} unit, inside the \textbf{QueueManager} component.
\begin{algorithm}\label{alg:retrievingOfATaxi}
\begin{algorithmic}
\REQUIRE $zone:$ a correct taxi zone
\ENSURE a provided taxi or an error message, if there is no taxi driver waiting in the queue \linebreak

\STATE $queues$: map having as keys the taxi zones and as value the corresponding queue \linebreak
$queue:= queues.VALUE(zone)$ \COMMENT {Retrieval of the queue relative to the provided taxi zone} 
$returnValue:$ return value
\IF {$queue.ISEMPTY()$}  
	\STATE{$returnValue:= error$} 
	\ELSE{
		\STATE $taxi:=queue.GETTAXI()$ \COMMENT{Takes the first taxi in the queue}
		\STATE $returnValue:=taxi$
	}
\ENDIF
\RETURN $returnValue$
\end{algorithmic}
\caption{Retrieval of a taxi from a queue}
\end{algorithm}
\paragraph{}From the pseudo-code we understand the following:
\begin{itemize}
	\item The \textbf{QueueContainer} stores the taxi queues in a map having has key the taxi zones
	\item Each queue is a particular data structure organized with a FIFO policy
\end{itemize}
\paragraph{} The following algorithm describe the adding of a taxi in a queue, action performed by the \textbf{ManagementLogic} unit using the interface provided by the \textbf{QueueContainer}
\begin{algorithm}
\caption{Adding of a taxi to a queue}
\begin{algorithmic} \label{alg:addingOfATaxi}
\REQUIRE{Inputs: \\
	\begin{itemize}
		\item $taxi:$ a taxi driver that must be inserted in a queue
		\item $zone:$ the zone in which the taxi driver wants to operate
	\end{itemize}
}
\ENSURE the adding of the taxi driver to the queue corresponding to the zone provided
\linebreak
\STATE $queues$: map having as keys the taxi zones and as value the corresponding queue \linebreak
$queue:= queues.VALUE(zone)$ \COMMENT {Retrieval of the queue relative to the provided taxi zone} \linebreak
$result = queue.ADDTAXI(taxi)$ \COMMENT{Adding the taxi driver at the bottom of the queue}
\RETURN
\end{algorithmic}
\end{algorithm}
\paragraph{}Lets analyze in more detail the data structure $queue$ which has the role of storing the taxi drivers during their waiting of calls: it has a FIFO (\textit{first-in-first-out}) policy for the management of its elements as it is visible in figure \ref{fig:FIFOpolicy}
\begin{figure}[H]
	\centering
	\includegraphics[scale=0.1]{../"Analysis Documents"/FIFO}
	\caption{FIFO policy}
	\label{fig:FIFOpolicy}
\end{figure}
In the Java programming language we could implement a class \texttt{TaxiQueue} which store an element of the class \texttt{LinkedList} which extends the interface \texttt{Queue}, which is already provided by the framework. Below you can see an example of usage of this class:

\lstinputlisting[language=Java,caption=Example of usage of the Queue interface in Java]{./AlgorithmDesign/QueueExample.java}

\section{Reservation Handler}\label{alg:reservationHandler}
The nature of the problem of managing the reservation waiting time before the submission of the request to a taxi driver requires a \textit{multi-threaded} solution.\\
We design the following algorithm to be the conjunction of behaviors of 3 different threads:
\begin{itemize}
	\item \textit{Reservation Receiver}
	\item \textit{Waiter}
	\item \textit{Reservation Forwarder}
\end{itemize}
\subsection{Reservation Receiver thread}
\begin{algorithm}[H]
\begin{algorithmic}
\REQUIRE Input: 
	\begin{itemize}
		\item $reservation$ (data structure containing: passenger data, meeting time and date, source and destination, number of passenger)
	\end{itemize}
Already present data structure:
\begin{itemize}
	\item $reservationList$: ordered list containing all the reservations waiting to be forwarded to a taxi driver. The order is done by meeting time.
\end{itemize}
\ENSURE The $reservation$ is added in the correct order in the $reservationList$ and the timer of the \textit{Waiter} is set correctly
\STATE $meetingTime = reservation.meetingTime$
\STATE $reservationList.PUT(reservation)$ \COMMENT{the reservation is put in order by meeting time} 
\STATE $index = reservationList.GETINDEX(reservation)$
\IF{$index$ is the first}
	\STATE Sends a signal to the \textit{Waiter} in order to stop is current timer (signal)
	\STATE $waitingTime=meetingTime - CURRENT\_TIME$
	\STATE Communicate to the \textit{Waiter} thread the $waitingTime$
\ENDIF
\end{algorithmic}
\caption{\textit{Reservation Receiver}: Adding of a reservation to the data structure}
\end{algorithm}
\subsection{Waiter thread}
\begin{algorithm}[H]
\begin{algorithmic}
\REQUIRE Already present data structure:
\begin{itemize}
	\item $timer$: is a counter that can be set as ON/OFF and which can raise an signal when the set time expires.
\end{itemize}
\ENSURE The correct setting of the timer
\STATE
\STATE $timer.count = 0$ \COMMENT{First initialization of the timer}
\WHILE{true}
	\IF{a signal is received}
		\STATE $waitingTime=$ the parameter communicated by \textit{Reservation Receiver}
		\STATE $timer.SETTIME(waitingTime)$
		\STATE $timer.START()$
	\ENDIF
	\IF{$timer$ expires}
		\STATE Send a communication to the \textit{Reservation Forwarder}
		\STATE $timer.STOP()$
	\ENDIF
\ENDWHILE
\end{algorithmic}
\caption{\textit{Waiter}: setting of the timer and timer termination}
\end{algorithm}
\subsection{Reservation Forwarder thread}
\begin{algorithm}[H]
\begin{algorithmic}
\REQUIRE Already present data structure:
\begin{itemize}
	\item $reservationList$: ordered list containing all the reservations waiting to be forwarded to a taxi driver. The order is done by meeting time.
\end{itemize}
\WHILE{true}
	\STATE Wait for a signal by the \textit{Waiter} thread
	\STATE A signal is received
	\STATE $first=$ the first index of $reservationList$
	\STATE $firstReservation=reservationList.GET(first)$ \COMMENT{Next reservation is got from the list}
	\STATE $reservationList.REMOVE(firstReservation)$ \COMMENT{Next reservation is removed from it}
	\STATE $nextReservation = reservationList.GET(first)$
	\STATE $waitingTime = nextReservation.meetingTime - CURRENT\_TIME$
	\STATE Communicate to the \textit{Waiter} thread the $waitingTime$
	\STATE Forwards the $firstReservation$ to the RequestQueueManager
\ENDWHILE
\end{algorithmic}
\caption{\textit{Reservation Forwarder}: Forwarding of a reservation to the RequestQueueManager}
\end{algorithm}

\paragraph{} Here we presented a pseudo-code of the behavior of this three threads. Each language, in particular Java, has its way to implement threads and waiting conditions. In this document we don't care about implementation.