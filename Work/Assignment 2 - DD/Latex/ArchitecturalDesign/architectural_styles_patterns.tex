\section{Architectural Styles and Patterns}

\subsection{Client - Server}
\paragraph{}The application main architecture uses the \textit{client-server} style.
\begin{figure}[H]
	\centering
	\includegraphics[scale=0.4]{../"Analysis Documents"/client-server}
	\caption{Interactions in a client-server architecture}
\end{figure}
The reasons for this choice are:
\begin{itemize}
	\item The \textit{centrality of the server} and the \textit{sparsity of the clients}: we have different users that need some type of service provided by a particular organization
	\item The \textit{absence of business logic client-side}: our user devices (web browser and mobile applications) must not need to know the back-end logic of the services they are invoking. This ensure scalability and the possibility to add new services or modify the already present with a minimum effort in the update of the applications client side.
\end{itemize}

\subsection{Three-tier-architecture}
We decided to use a \textit{three-tier-architecture} for our system.\\
A schematic representation of this responsibility distribution is at figure \ref{fig:3-tier-architecture}

\begin{figure}[H]
	\centering
	\includegraphics[scale=0.5]{../"Analysis Documents"/3-tier-architecture.png}
	\caption{A representation of a three-tier-architecture}
	\label{fig:3-tier-architecture}
\end{figure}
\paragraph{} So we subdivide our system in three extremely separated layers (or \textit{tiers}):
\begin{itemize}
	\item A \textit{presentation} layer for the graphic rendering of the data and events generated by the system, and from which the end user can interact with the system
	\item An \textit{application} layer entitled to manage all the business logic of the system
	\item A \textit{Data} layer responsible of the storage of informations to be used by the application and presentation layer
\end{itemize}
The three levels are completely independent and can be replaced easily. In particular, the presentation layer cannot communicate directly with the data tier, but it must forward its requests to the application one, which will use some data access framework to access the database.


