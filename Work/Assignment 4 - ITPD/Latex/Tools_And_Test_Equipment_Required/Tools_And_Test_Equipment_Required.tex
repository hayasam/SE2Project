\chapter{Tools and test equipment required}
\section{JUnit}
\textit{Junit} is the framework of choice for the evaluation and statistical analysis of the test results. The reasons for this choice are:
\begin{itemize}
	\item It is the most famous testing framework, and there are good chances that it is well known in the development team devoted to the project.
	\item It uses the Java programming language, and since we are planning to use the JEE environment, its integration will be simpler
	\item There is a graphical plug-in for the Eclipse IDE, which will be the programming environment of choice during the development.
\end{itemize}

\subsection{DbUnit}
In addition we can use also use \textit{DbUnit}, which is a JUnit extension that we can use to initialize the database with a "state" that is well known. The real advantage of this approach is that we divide the test data creation from the test code.

\section{Versioning system}
We believe that the usage of a versioning system (like \textit{Git}) is essential for the testing phase success, as for the development phase. Developers will provide corrections and discuss enhancements, with the side effect of creation of a lot of intermediate different versions of the same unit or set of units.
