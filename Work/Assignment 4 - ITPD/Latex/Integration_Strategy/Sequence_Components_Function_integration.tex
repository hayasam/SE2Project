\section{Sequence of Component/Function integration}
We have only one sub-system, so we proceed directly to its integration sequence
\begin{figure}[H]
	\centering
	\includegraphics[scale = 0.6]{"../Analysis Documents/testSequence"}
	\caption{Components of the system and their dependencies}
	\label{fig:components}
\end{figure}
\paragraph{}In figure \ref{fig:components} we show the components of our system as we modeled it in the Design Document. In particular, we show the \textit{dependencies} between them through arrows with this notation: if a component A points to a component B, it means that A \textit{needs} B for its execution. For example: the TaxiManager component needs for its execution components GeographicEngine and QueueManager.

\paragraph{} This is the designed integration plan for the components:
\begin{center}
	\begin{tabular}{ l | l | l }
	\textbf{ID} & \textbf{Components interaction} & References \\ \hline
	I1 \label{I1} & RideManager, TaxiManager, QueueManager $\implies$ GeographicEngine & \\ \hline
	I2 \label{I2} & TaxiManager $\implies$ QueueManager & \\ \hline
	I3 \label{I3} & RideManager $\implies$ TaxiManager & \\ \hline
	I4 \label{I4} & Passenger, TaxiDriver $\implies$ AccountManager & \\ \hline
	I5 \label{I5} & Passenger $\implies$ RideManager & \\ \hline
	I6 \label{I6} & TaxiDriver $\implies$ TaxiManager & \\
	\end{tabular}
\end{center}
%\subsection{Software integration sequence}
%\subsection{Subsystem integration sequence}
